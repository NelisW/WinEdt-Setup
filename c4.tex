% -*- TeX -*- -*- UK -*- -*- Soft -*-"

\chapter{Installing LZMA-Archive  Packages on MikTex}
\label{sec:miktexpackages}

\section{Problem}

MikTeX uses the LZMA compression tool to compress tar package files, such as \verb"fancyhdr".  For some reason my implementation of MikTeX 2.5 and 2.7 are unable to read its own compressed files (e.g. \verb"fancyhdr.tar.lzma"). This means that the package manager is unable to install such files.


\section{Solution}

On
\verb"http://comments.gmane.org/gmane.comp.tex.miktex/6578"
 Christian Schenk writes:
{\small
\begin{verbatim}
Re: Format of new LZMA archives

> just a technical question: Is there something special about the format
> of the new .lzma archives? When I try to unpack those files with 7z
> (v4.47 beta) it gives me an error (Can not open file as archive).

The .tar.lzma files in the MiKTeX package repository were created with
lzma.exe from the LZMA SDK (http://www.7-zip.org/sdk.html). You can run

lzma d PACKAGE.tar.lzma -so | tar -xvf

to extract files from a package.
\end{verbatim}
}

In order to install packages archived with LZMA I had to resort to the following procedure:

\begin{enumerate}
\item
Download the LZMA SDK as Schenk advises from \verb"http://www.7-zip.org/sdk.html"

\item
Rename \verb"PACKAGE.tar.lzma" to \verb"PACKAGE.lzma", where PACKAGE is the package name.

\item
Unzip the archive to a tar file with the following command

\verb"lzma d PACKAGE.lzma PACKAGE.tar"

\item
Untar the tar-ball with  Winzip or whatever tool you are using, retaining the directory structure.

\item
Copy the directory structure to the MikTek directory (e.g. \verb"C:\Program Files\MiKTeX" 2.7). Take care to copy the untarred files to the appropriate directories in the MikTek directory structure.


\end{enumerate}


